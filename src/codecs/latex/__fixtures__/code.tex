% This is a test fixture focussed on Stencila Schema `Code` node types.
%
% It's main purpose is to see how we can represent these types of
% nodes in LaTeX (i.e what LaTeX command Pandoc parsers and how it
% parses them).
%
% See https://github.com/stencila/schema/tree/master/schema for the
% semantics and roles of these various types.
%
% See https://github.com/jgm/pandoc/blob/master/src/Text/Pandoc/Readers/LaTeX.hs
% for how Pandoc parses LaTeX.

% A `CodeFragment` with no `programmingLanguage` can be represented
% using `\verb`.
Some inline \verb|code|.

% A `CodeFragment` with a `programmingLanguage` can be represented
% using `\lstinline`.
Some inline Java \lstinline[language=Java]{public class MyClass}.

% A `CodeBlock` with no `programmingLanguage` can be represented
% using a `verbatim` environment.
\begin{verbatim}
This is like the HTML <pre> element
\end{verbatim}

% A `CodeBlock` with a `programmingLanguage` can be represented
% using a `lstlisting` environment.
\begin{lstlisting}[language=js]
const foo = 'bar'
console.log(foo)
\end{lstlisting}

% The above are all "static" code nodes with standard approaches for
% inclusion in LaTeX. The following a "dynamic" code nodes - where we
% want to find was to represent them in LaTeX using modifications of
% syntaxes that Pandoc already knows about.

% A `CodeExpression' could be represented by adding `exec` to
% the `\lstinline` language. It would be preferable to use a comma between them
% (i.e. a different Pandoc "attr" on the element) but at the time of writing
% Pandoc only seems to pass `language` and not other attributes
An evaluated Python expression \lstinline[language=python exec]{pow(2,2)}.

% A `CodeChunk' could be represented by adding the `exec` class
% to `\lstlisting` language.
% In this case, Pandoc v2.12 does parse "exec" as a separate attribute on the
% element (in contrast to lstinline) but for consistency we don't use the comma here.
\begin{lstlisting}[language=r exec]
plot(y~x, data)
\end{lstlisting}

% A `CodeChunk` with `id` and `caption` properties can be represented
% using a lstlisting as well (but might require a second parsing of the
% caption string)
\begin{lstlisting}[language=r exec, caption={
    A plot of the data which hasd a mean of \lstinline[language=r exec]{ mean(data) }
}, label=fig2]
plot(data)
\end{lstlisting}
