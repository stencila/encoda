\documentclass[oneside]{article}

\usepackage{graphicx} % for figures
\usepackage{color} % for colored text
\usepackage{float} % for forcing figure placement
\usepackage{enumerate} % for bullet point lists
\usepackage{setspace} % line spacing
\usepackage{microtype} % slightly tweak font spacing for aestheticsgins
\usepackage{booktabs} % horizontal rules in tables
\usepackage{amsmath} % text in equations
\usepackage{blindtext}
\usepackage[breakable]{tcolorbox} % text box
\usepackage[margin=1in]{geometry} % margins
\usepackage[sc]{mathpazo} % Palatino font
\usepackage[T1]{fontenc} % use 8-bit encoding that has 256 glyphs
\usepackage[small,labelfont=bf,up,up]{caption} % custom captions under/above floats in tables or figures

\usepackage[shortlabels]{enumitem} % customized lists (shortlabels
                                % necessary to have i., ii., etc., in enumerate)
\setlist[itemize]{noitemsep} % Make itemize lists more compact

\usepackage{abstract} % customization of abstract
\renewcommand{\abstractnamefont}{\normalfont\bfseries} % set the "Abstract" text to bold
\renewcommand{\abstracttextfont}{\normalfont\small\itshape} % set the abstract itself to small italic text
\usepackage{titling} % customization of title section
\usepackage{titlesec} % customization of titles

\usepackage{fancyhdr} % headers and footers
\pagestyle{fancy} % all pages have headers and footers
\fancyhead{} % blank out the default header
\fancyfoot{} % blank out the default footer
\fancyhead[C]{Authors et al. $\bullet$ August 2100 $\bullet$ an{\color{red}R}$\chi$ve} % custom header text
\fancyfoot[RO,LE]{\thepage} % custom footer text

\usepackage[hidelinks]{hyperref} % For hyperlinks in the PDF

\usepackage{natbib}
\bibliographystyle{apalike}

\setlength\columnsep{20pt}

%----------------------------------------------------------------------------------------
%	TITLE SECTION
%----------------------------------------------------------------------------------------

\setlength{\droptitle}{-4\baselineskip} % move the title up

\pretitle{\begin{center}\Huge\bfseries} % article title formatting
\posttitle{\end{center}} % article title closing formatting
\title{A title here} % article title
\author{\textsc{An author here$^{1,2*}$}, \textsc{Another author
    here$^{1*}$}, \\ \textsc{Yet another author here$^{1*}$},
  \textsc{Last author here$^{1*}$} \\
\small $^1$Stencila School of Reproducible Science, Kaikoura, New Zealand\\
\small $^2$Department of Reproducibility, Kaikoura, New Zealand\\
\small
\href{mailto:someone@stenci.la}{Corresponding authors$^*$: someone@stenci.la,}
\href{mailto:anothersomeone@stenci.la}{anothersomeone@stenci.la}
%\and % Uncomment if 2 authors are required, duplicate these 4 lines if more
%\textsc{Jane Smith}\thanks{Corresponding author} \\[1ex] % Second author's name
%\normalsize University of Utah \\ % Second author's institution
%\normalsize \href{mailto:jane@smith.com}{jane@smith.com} % Second author's email address
}
\date{\today} % Leave empty to omit a date
\renewcommand{\maketitlehookd}{%
\begin{abstract}
  \noindent This is an example abstract.
  In the abstract, scientists will write short sentences summarizing
  the motivation for the paper, briefly describing the methods, and
  more importantly, explaining how their results push the field
  forward.
\end{abstract}
\centering [Keyword 1, keyword 2, keyword 3, key-word 4]
}

%----------------------------------------------------------------------------------------

\doublespacing

\begin{document}

% Print the title
\maketitle

%----------------------------------------------------------------------------------------
%	ARTICLE CONTENTS
%----------------------------------------------------------------------------------------

\section{Section: Introduction}

Let us try to generate some errors here.
First, I am going to try to cite some papers with 'e.g.' as arguments
to the \texttt{\symbol{92}citep\{\}} function from ``natbib'' \LaTeX package, like
this:
\citep[e.g.,][]{article1,article2,article3}.
Note that we are typing one sentence per line -- Emacs might still
line wrap, but we never have a period ``.'' followed by a word on the
same line.

Here is an inline equation: $\text{P}(\theta|D) =
\frac{\text{P}(D|\theta)\text{P}(\theta)}{\text{P}(D)}$.

\subsection{Subsection: A table and a figure}

We are now going to place a table here:

\begin{table}
\begin{center}
\begin{tabular}{lc}
\hline
$i$ & $j$ \\
\hline
0 & 0.0 \\
1 & 0.1 \\
2 & 0.2 \\
3 & 0.3 \\
4 & 0.4 \\
\hline
\end{tabular}
\end{center}
\caption{Let us type a very long caption. Sometimes journal styles will
  fail to nicely format a very long caption on top of a narrow table.
\label{tab:tab1}}
\end{table}

And now we will refer to Table \ref{tab:tab1}.

Time to try a figure:

\subsubsection{Subsubsection: Equations}

We will try to type a simple equation below:

\begin{equation}
  \text{E}[x] = \frac{1.0}{1.0}
  \label{eq:eq1}
\end{equation}

And now another one inside a box (\texttt{\symbol{92}tcolorbox[a few
  args]\{\}}) environment:

\begin{tcolorbox}[breakable, width=\textwidth, colback=gray!10, boxrule=0pt,
  title=Box 1: A box title fonttitle=\bfseries]
  \small

  Inside this box, we will put another equation that we got from
  \citealt{article2}:

  \begin{equation}
    x\mathbf{T} = 1.0
    \begin{bmatrix}
      1 & 2 & 3\\
      4 & 5 & 6\\
      0 & 0 & 6
    \end{bmatrix}
    \label{eq:eq2}
  \end{equation}

  Hopefully we will be able to see this box environment.
\end{tcolorbox}

\subsection*{Subsection*: Figures}

This subsection title should not be numbered.
We will place a .png figure now:

\begin{figure}[ht!]
  \includegraphics[width=4cm]{image.png}
  \caption{This is a screenshot from the .svg figure
    downloaded from Stencila's webpage (in the press
    section).
  We are making the logo very narrow and typing a huge
  caption to see what happens.}
  \label{fig:pnglogo}
\end{figure}

% And we can now refer to the .png logo (Fig. \ref{fig:pnglogo}).

\section*{Section*: vast amounts of text}

\blindtext[4]

\bibliography{kitchen-sink}

\end{document}
